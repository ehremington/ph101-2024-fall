At the end of this worksheet you should be able to  
\begin{itemize}
	\item find the momentum of an object or collection of objects.
	\item find the change in momentum of an object or collections of objects due to an impulse.
	\item use the conservation of momentum to solve for an unknown quantity.
	\item use the principle of relative velocity to solve for an unknown in elastic collisions.
\end{itemize}

Note that at in every problem this week we will be ignoring friction and air resistance. Its not that momentum can't work with those quantities, but momentum does not really help tell us anything new about them, so our problems will not involve friction unless explicitly specified. When using the conservation of momentum in this way, we really restrict ourselves to talking about the motion of the objects \emph{just before} they collide as well as \emph{just after} they have stopped colliding.

\begin{enumerate}
\setlength\itemsep{1 in}

\item In this problem a bug and car collide. Assume the car is coasting frictionlessly. Also assume you can measure quantities with perfect accuracy. 
\begin{itemize}
	\item A \SI{1}{\gram} bug is flying east through the air along a road at \SI{5}{m/s}. What is the bug's momentum? Remember that momentum is a \emph{vector} so if you are calling the momentum positive that means east.\bigskip
	\item A \SI{1000}{kg} car is traveling west along this same road at a speed of \SI{20}{m/s}. What is the car's momentum. Remember the sign!\bigskip
	\item What is the total momentum of this system?\bigskip
	\item When the bug collides with the windshield of the car, what is the momentum of the bug-car system? What is the total mass of the car now? What is the speed of the car?
\end{itemize}



\item You want to close a door but you don't want to get up. You look around and see that there is a bouncy ball that looks like you could throw it pretty fast. For some reason there is also a wad of clay that you know would stick to the door if you threw it. The ball and the clay have the same mass. Which one should you throw against the door to close it most effectively?\\
Some starters:
\begin{itemize}
	\item Choose a mass for your ball and clay, and choose a velocity. Or leave it as $m$ and $v$ and work it \emph{in general}.
	\item If the ball hits the door and bounces back perfectly, its speed should be basically the same after it bounces off the door. How has its velocity changed? How much has its momentum changed?\bigskip
	\item If the clay hits the door and sticks, then what is its change in momentum? Its speed is basically zero after the collision.\bigskip
	\item Which one of these cases would close the door more effectively? If the time interval of the collision is about the same for both of these cases, and we model the force of collision as constant, then what is the ratio of the force from the bouncy ball, to the force of the clay wad?
\end{itemize}


\item You know the kinetic energy of a \SI{10}{kg} object is \SI{100}{\joule}, then what is the momentum of the object? Now do this \emph{inside out}. Now do it \emph{in general}. In other words show where $K=\dfrac{p^2}{2m}$ comes from.

\item
\begin{itemize}
	\item If I am are standing motionless, what is my momentum? I have a mass of \SI{75}{kg}.
	\item What if I am wearing a backpack with 100 - \SI{0.5}{kg} baseballs while standing on ice (with skates)?
	\item I throw one of the baseballs with a velocity at \SI{-15}{m/s}. What is its momentum?\bigskip
	\item What is my mass after I throw it (think about the mass of my backpack)? What is my momentum after I throw the baseball? What is my velocity?\bigskip
	\item If I throw another baseball with a velocity of \SI{-15}{m/s} \emph{relative to my current velocity}, then what is the velocity of the ball relative to the ice?\bigskip
	\item What is my new velocity after throwing this second baseball? How much did my velocity change from before?\bigskip
	\item If I did this again, throwing a baseball with a velocity of \SI{-15}{m/s} \emph{relative to my current velocity}, what is my velocity after I throw it?\bigskip
	\item I keep doing this until I am out of baseballs. How fast am I going?
\end{itemize}

\item An empty \SI{10}{kg} wagon is rolling past me at a speed of \SI{5}{m/s} while I am holding a \SI{30}{kg} bag of concrete. If I drop the concrete bag into the wagon right at it gets to me, what is its speed immediately after that?\bigskip

\item 
I drop a \SI{1}{kg} basketball from a height of \SI{1}{\meter} and it bounces off the floor and rises to a height of \SI{0.75}{\meter}. How much energy was converted to internal energy?\\
Some starters:
\begin{itemize}
	\item What is the basketball's velocity right before it hits the ground?
	\item What is the basketball's velocity right after it hits the ground? Think about what it must be for it to rise to \SI{0.75}{\meter}.
	\item What is the ratio of the change in kinetic energy?
	\item What is the change in kinetic energy?
	\item What is the change in momentum of the ball?
	\item Is momentum conserved here? Why or why not?
	\giantskip
\end{itemize}

\item Explosions are actually a kind of collision, but in reverse. Let's work a problem to see. Two indestructible objects are tied together with a stick of dynamite between them and everything is at rest. When it explodes, they fly away from each other in opposite directions. One of the objects has a mass of \SI{10}{kg} and the other a mass of \SI{3}{kg}. The \SI{10}{kg} object flies away with a speed of \SI{100}{m/s}. What is the velocity of the \SI{3}{kg} object? At least how much energy was the explosion (some of it was probably also put into internal energy and sound)? What other problem in this worksheet is this similar to?


\item A \SI{5}{kg} gun fires a \SI{10}{\gram} bullet at a velocity of \SI{500}{m/s}. The explosion that causes this shot takes place in \SI{1}{millisecond}. What force is exerted on the gun and what is the gun's recoil velocity?

\item Two objects collide elastically, one has a mass of \SI{1}{kg} and the second has a mass of \SI{3}{kg}. The first object is traveling to the right at a speed of \SI{3}{m/s} and the second is traveling in front of it initially at a speed of \SI{1}{m/s}. What is the velocity of the objects after the collision. Think of and work as many variations of this problem as you can.
\giantskip

\item Two objects collide elastically. Two objects collide elastically, one has a mass of \SI{1}{kg} and the second has a mass of \SI{3}{kg}. The first object is traveling to the right at a speed of \SI{3}{m/s}. With what velocity would the second object need to be travelling so that after the collision, the first object was motionless?


\item
A massive object moving with a velocity $v$ and is going to collide elastically with a very small object that is initially at rest. What is the velocity of these two objects after they collide? (\emph{By massive and very small I mean when you add or subtract m$_1$ and m$_2$ they are indistinguishable from m$_1$ due to the rules of significant figures.})

\item 
Reverse the previous problem. A tiny object is colliding with velocity $v$ with a huge object that is at rest. What is the final velocity of the two objects?

\item Let's work the ballistic pendulum problem like we do in the lab and do it \emph{in general}. So we have a bullet of initial velocity $v_{bi}$ and mass $m_b$ and a target initially at rest with mass $m_t$. The collision is perfectly inelastic, and the pendulum (with the bullet inside) rises to a height $h$ after the collision takes place. What is the initial velocity of the bullet?\giantskip

\item Now in question \#2 of the lab write-up it asks could this experiment be done with an \emph{elastic} collision, given that we only measure the same things as we did in the inelastic collision case. So same parameters as before, but this time when the ball bounces off the target it has a speed $v_{bf}$ and the pendulum rises to a height $h$ but this time without the bullet embedded in it.\giantskip

\end{enumerate}