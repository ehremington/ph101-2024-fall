At the end of this worksheet you should be able to  
\begin{itemize}
%	\item apply the concepts of vector components to velocity.
%	\item use Newton's 1st and 2nd laws to discuss the motion of objects in 2 dimensional projectile motion.
%	\item apply the kinematic equations to be able to completely describe the motion of a projectile.
	\item apply the relationships between angle and motion at the edge of a circle to describe the motion of an object in circular motion.
	\item apply Newton's 2nd law in the radial direction to solve interesting problems involving motion of objects in a circular path.
	\item apply the principles of radial net force and circular motion to planetary orbits and satellites as well horizontal and vertical paths near earth's surface.
\end{itemize}

\begin{enumerate}
	\setlength\itemsep{2 in}
%
%	\item I initially throw a baseball at an angle of \ang{10} with respect to horizontal. The initial speed of the ball is 60 miles per hour. What is the x- and y- component of the initial velocity?
%	
%	\item If the x-component of a velocity vector is \SI{+10}{\meter/\second} and the y-component is \SI{+20}{\meter/\second}, then what is the speed of the ball and what is the angle of its initial trajectory with respect to the horizontal?
%	
%	\item A baseball is thrown and then lands with a speed of \SI{25}{m/s} at an angle of \ang{30} degrees to the horizontal. If I described this as an angle with respect to the vertical what would I have said. What is the x- and y-components of the final velocity of this ball?
%	
%	\item In the last problem, what is the initial horizontal velocity of the baseball? What information would you need in order to find what the initial vertical velocity component was?
%	
%	\item I throw a baseball with a speed of \SI{40}{m/s} at an angle of \ang{60} degrees above the horizontal. The ball is released at a height of 1.6 meters. For each second that goes by, calculate the horizontal velocity, the vertical velocity, the horizontal position and the vertical position.
%	
%	\begin{tabular}{lllll}
%		\hline
%		time & x-vel & y-vel & x-pos & y-pos \\
%		\hline
%		\hline
%		\SI{0}{\second}&&&\SI{0}{\meter}&\SI{1.6}{\meter} \\
%		1& \\
%		2\\
%		3\\
%		4\\
%		5\\
%		6\\
%		7\\
%		8\\
%		
%	\end{tabular}
%	
%	\item Plot your data for the last problem vs. time. ($v_x$ vs $t$, $v_{y}$ vs $t$, $x$ vs $t$, $y$ vs $t$) \hugeskip
%	
%	\item I throw a baseball with a speed of \SI{40}{m/s} at an angle of \ang{60} degrees above the horizontal. The ball is released at a height of 1.6 meters. What is the vertical displacement when the baseball hits the ground? When will the ball reach its maximum height, and when will it hit the ground? 
%	
%	\item For a soccer ball that is kicked horizontally off the edge of a \SI{10}{\meter} cliff with an initial speed of \SI{20}{m/s}, what are the x- and y- components? When does the ball land? \bigskip
%	
%	\item If I just dropped the soccer ball off the \SI{10}{\meter} cliff, how long would that take to land? Compare this answer to the previous problem. Is that surprising? What if you kicked the ball with \SI{100}{m/s} initial speed horizontal velocity? Surely that would matter...\bigskip 
%	
%	\item OK back to the soccer ball that is kicked horizontally off the edge of a \SI{10}{\meter} cliff with an initial speed of \SI{20}{m/s}. How far from the base of the cliff will it land?
%	
%	\item For the previous problem, what is the \emph{speed} of the ball when it lands and at what angle with respect to the horizontal does it hit the ground?
%	
%	\item Let's do the last problem inside out. So take how far away from the base of the cliff the soccer ball lands, and work backwards to find its initial horizontal speed. The only assumption is that it is initially kicked exactly horizontally.\bigskip
%	
%	\item
%	Now let's do the previous problem \emph{in general}. If I kick a ball directly horizontally off a cliff of height $h$, and its lands a distance $x$ away from the base of the cliff, then what was the initial velocity of the ball? (We will use this result in lab soon.)
	
	\item How many degrees are in \SI{1}{\radian}?
	\item A soccer ball of radius \SI{10}{cm} spins through an angle of \ang{20}, then how many radians is that? What distance has a point on the equator of the ball traveled? What if it spins through \ang{750}, then what distance has a point on the edge traveled?
	
	\item When you roll something along the ground, it is spinning of course, but it is also moving linearly (its center of mass is moving). It turns out that the distance the edge of a soccer ball moves as it spins is equal to the linear distance the ball moves, as long as it does not slip. So if a soccer ball of radius \SI{10}{cm} rolls at constant angular speed through an angle of \SI{500}{\radian}, then how far has it rolled? If it takes 10 seconds to do this, what was its angular speed and what was its linear speed?
	
	\item When a car turns at constant speed, it travels along an approximately circular path. In which direction does the net force act and what provides this net force?
	
	\item For a \SI{1000}{kg} car turning like in the previous problem, if the coefficient of friction between the tires and the road is $\mu=0.5$, then what is the maximum static force of friction that the road could provide to the car? If the car is going around a bend of radius \SI{50}{\meter}, how how fast could it go around the bend without sliding?
	
	\item If the same \SI{1000}{\kilogram} car is attempting to go around a bend of radius \SI{20}{\meter}, at \SI{20}{m/s}, then can it do this safely without sliding? ($\mu = 0.5$ still)
	
	\item The earth orbits the sun, and while its path around the sun is not exactly circular, its close enough to treat that way here. What is the angular velocity of the earth around the sun? To do this, think about how long it takes to go one full revolution around the sun. How many radians is a revolution? So now how many radians per second does the earth travel around the sun? 
	\item What is the radius between the earth and the sun? (look this up in your book or google) Using the answer from the previous problem, what does this mean for the \emph{tangential speed} of the earth around the sun?
	\item Now without looking it up, how could we use this information to determine the mass of the sun?  The formula for the force of gravity between to masses can be written as, $F_g=\dfrac{G m_1 m_2}{r^2}$ ($G=\SI{6.67e-11}{\newton \meter^2/\kilogram^2}$). Note that this not the form of the force gravity that we have been using. Why is that? Now look up the mass of the sun and see how how close we got.
	\item By the way, how can we use free fall to get a measure of the mass of the earth? If we got to the lab and measure an acceleration of a \SI{1}{\kilogram} mass to be \SI{9.82}{m/s^2}, then how can we calculate the mass of the earth?
	
	\item In order to put a satellite into orbit around the earth, it needs to be traveling at a specific distance with a specific velocity, otherwise the force of gravity from the earth may be too large, and it will crash, or too small and it will fly away into space. So suppose you wanted to put a \SI{1000}{kg} satellite in orbit around the earth at a distance of 1000 km above the \emph{surface of the earth}. How fast would this satellite need to be going in order to have this orbit?
	
	\item If you wanted to kick a soccer ball horizontally off a cliff and have it go into orbit near the surface of the earth, then what velocity would you need to give it to achieve this?
	
	\item A pendulum is swinging in a horizontal circle. The length of the pendulum is \SI{1}{m}. If the angle of the pendulum string is \ang{25}, then what is the radius of travel of the pendulum bob?
	
	\item The mass of the pendulum bob from the previous problem is \SI{1}{kg}. What upward force is necessary to keep the pendulum from moving up and down? What does this imply about the tension in the string? What does this mean for the radial tension force? How fast must this pendulum bob be moving?
	
	\item When you are swinging a ball at the end of a string in a \emph{vertical} circle, explain why the tension in the string is higher when the ball is at the bottom of its path, than when it is at the top of its path.
	
	\item A roller coaster cart is doing a loop-the-loop. When the cart is at the top, what forces are acting on the cart to keep it in its circular path? What is the minimum force that would still technically mean that the cart is still in contact with the track? For a \SI{30}{m} radius loop, what is the minimum speed that the cart must be going to make the loop without losing contact with the track?   
	
\end{enumerate}